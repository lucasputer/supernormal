\documentclass[10pt,mathserif]{beamer}%fleqn
\usepackage{amscd, amsthm,amssymb,latexsym}
\usepackage{xcolor}
\usepackage[utf8]{inputenc}
\usepackage[spanish]{babel}
%\usepackage{beamerthemesplit}


\usepackage{url}
\mode<presentation>{
	\usecolortheme{}
	\useinnertheme{}
	\useoutertheme{}
}

%\usepackage{enumitem}
%\setenumerate{itemsep=0.005ex,topsep=3pt}
%\setitemize{itemsep=0.005ex,topsep=3pt}

\expandafter\def\expandafter\insertshorttitle\expandafter{%
  \insertshorttitle\hfill%
  \insertframenumber\,/\,\inserttotalframenumber}


\setbeamertemplate{headline}{}
\beamertemplatenavigationsymbolsempty

%\definecolor{magenta}{RGB}{250,0,150}
%\definecolor{black}{RGB}{250,0,150}

%\definecolor{mora}{RGB}{120,0,100}
%\definecolor{pink}{RGB}{255,100,160}

%\setbeamercolor{block title example}{bg=white,fg=red}
%\setbeamercolor{structure}{bg=white,fg=red}
%\setbeamercolor{title}{bg=white,fg=red}
%\setbeamercolor{normal text}{bg=white,fg=mora}
%\setbeamercolor{frametitle}{bg=white,fg=red}

\begin{document}

%\small

\binoppenalty=10000 
\relpenalty=10000
\hyphenpenalty=5000
\exhyphenpenalty=1000


\newtheorem{observation}{Observation}
\newtheorem{conjecture}{Conjecture}
\newtheorem{belief}{Belief}
\newtheorem{question}{Question}



\newcommand{\Z}{{\mathbb{Z}}}
\newcommand{\N}{{\mathbb{N}}}
\newcommand{\Q}{{\mathbb{Q}}}
\newcommand{\R}{{\mathbb{R}}}
\newcommand{\floor}[1]{\lfloor #1 \rfloor } 
\newcommand{\ceil}[1]{\lceil #1 \rceil }
\newcommand{\abs}[1]{\left| #1 \right|}
\newcommand{\card}{\mbox{\raisebox{.13em}{{$\scriptstyle \#$}}}}
\newcommand{\expa}[1]{\{#1\}}
\newcommand {\base}[2]{\langle{#1};{#2}\rangle}
\newcommand{\ybar}{{\overline{y}}}
\newcommand{\xbar}{{\overline{x}}}

\newcommand{\cf}{\text{\em cf}}
\newcommand{\eps}{\varepsilon}
\newcommand{\wh}[1]{\widehat{#1}}
\newcommand{\NN}{\mathbb{N}}
\newcommand{\RR}{\mathbb{R}}
\newcommand{\uno}{\mathbbm{1}}


\newcommand{\alocc}[2]{|\!|#1|\!|_{#2}}
\newcommand{\occ}[2]{|#1|_{#2}}

\languagepath{spanish}
\deftranslation[to=spanish]{Theorem}{Teorema}
\deftranslation[to=spanish]{theorem}{teorema}
\deftranslation[to=spanish]{Definition}{Definición}
\deftranslation[to=spanish]{definition}{definición}
\deftranslation[to=spanish]{Problem}{Problema}
\deftranslation[to=spanish]{problem}{problema}
\deftranslation[to=spanish]{Corollary}{Corolario}
\deftranslation[to=spanish]{corollary}{corolario}
\deftranslation[to=spanish]{Lemma}{Lema}
\deftranslation[to=spanish]{lemma}{lema}




%\title[Del azar con dos símbolos al azar con tres símbolos]{Del azar con dos símbolos al azar con tres símbolos}
%\author[Ariel Zylber]{Ariel Zylber}
%\institute{Universidad de Buenos Aires, Argentina}
%\date{\vspace*{-3cm}\footnotesize{Tesis de Licenciatura, Universidad de Buenos Aires, Noviembre 21, 2017}}

\title{{\normalsize Tesis de Licenciatura en Ciencias de la Computación}
\\\vspace*{2cm}
\mbox{\Large Números Muy Normales}}

\author{\large Lucas Puterman
\\\vspace*{1cm}}


\date{{\footnotesize
\hspace*{-6cm}
\begin{tabular}{l}
Directora: Ver\'onica Becher \\
Departamento de Computaci\'on\\
Facultad de Ciencias Exactas y Naturales\\
 Universidad de Buenos Aires\\
20 de Noviembre, 2019
\end{tabular}
}}

\begin{frame}
\maketitle
\setcounter{framenumber}{0}
\thispagestyle{empty}
\end{frame}



\begin{frame}
\frametitle{Sobre secuencias aleatorias}

\begin{center}
  \includegraphics[scale=0.5]{imagenes/peso.jpg}
\end{center}

Supongamos que tiramos una moneda infinitas veces y anotamos un $1$ cada vez que sale cara y $0$ cada vez que sale ceca ¿Cuáles de estas secuencias es creíble que sea el resultado de este experimento?
\pause
\bigskip

\begin{itemize}
\item $11111111111111111111111111111111111\ldots$
\pause
\item $01001000100001000001000000100000001\ldots$
\pause
\item $01010101010101010101010101010101010\ldots$
\pause
\item $10000110001010001110010010110011010\ldots$
\end{itemize}
\end{frame}


\begin{frame}
\frametitle{Secuencias normales}
Podemos pensar que en una secuencia aleatoria no hay ningún patrón de~$\ell$ símbolos que sea más frecuente que otro.\\
\pause
\begin{definition}
Notamos $\occ{u}{v}$ a la \textit{cantidad de ocurrencias} de la palabra $v$ dentro de la palabra $u$.\\
\pause
\medskip
Además, notamos $u[i,j]$ a la subsecuencia de $u$ formada tomando todos los símbolos entre el $i$ y el $j$ inclusive.
\end{definition}
\pause
\medskip
Nos gustaría que para un prefijo de una secuencia aleatoria suficientemente grande, 
la cantidad de ocurrencias de cada palabra de cierta longitud sea casi la misma.
\end{frame}


\begin{frame}
\frametitle{Secuencias normales}

\begin{definition}[{{\scriptsize  Borel, 1909}}]
Dado un alfabeto $A$ y alguna secuencia infinita $u \in A^{\omega}$, 
decimos que $u$ es \textit{simplemente normal para la longitud $\ell$} 
si cada secuencia $v$ de longitud 
%$l$
\color{magenta}
$\ell$
\color{black} verifica que 
$$
\lim_{n \rightarrow \infty} \frac{\occ{u[1,\ell n]}{v}}{n} = \frac{1}{|A|^{\ell}}.
$$
\medskip
\pause

\medskip
Decimos que $u$ es \textit{normal} 
si es simplemente normal para toda longitud~$\ell \in \NN$.
 \end{definition}
\pause

\begin{problem}[{{\scriptsize  Borel, 1909}}]
Encontrar ejemplos naturales de secuencias normales. \\
Decidir si la representación en base $b$ de $\pi$, $e$ ó $\sqrt{2}$ es normal.
\end{problem}
\end{frame}


\begin{frame}
\frametitle{La secuencia de Champernowne}

  \begin{problem}
  Encontrar algún ejemplo explícito de una secuencia normal.
  \end{problem}
  \pause
  \begin{theorem}[{{\scriptsize  Champernowne, 1933}}]
  La secuencia
  $$1234567891011121314151617181920\ldots$$
  es normal 
  sobre 
  el alfabeto $A = \{0, 1, \ldots, 9\}$.
  \end{theorem}
\end{frame}

\begin{frame}
\frametitle{$champ$, La secuencia que usaremos}
  

  \begin{theorem}[{{\scriptsize  Bugeaud, 2012}}]
    Sea $A$ un alfabeto. Llamamos  $X(n)$ a la concatenación de todas las palabras de lomgitud $n$ formadas por símbolos de $A$ en orden lexicográfico.
    
    La palabra infinita $X(1)X(2)\dots$ es normal en el alfabeto $A$
  \end{theorem}

  \medskip
  \pause
  En particular, nosotros vamos a usar el alfabeto  $A=\{0,1\}$ Entonces, por ejemplo:
  $$X(2) = 00 \: 01 \: 10 \: 11$$
  \medskip
  \pause
  Entonces, los primeros símbolos de la secuencia que llamamos $champ$ son:
  $$champ = 0 \: 1 \: 00 \: 01 \: 10 \: 11 \: 000 \: 001 \: 010 \: 011 \: 100 \: 101 \: 110 \: 111 \: 0000 \: 0001 \: \dots$$


\end{frame}

\begin{frame}
\frametitle{Supernormalidad}
Sea $x$ una secuencia binaria. Sea $A^\lambda_{k,n}(x)$  la frecuencia de ocurrencia de las palabras de longitud $n$ que ocurren exactamente $k$ veces comenzando en las primeras $\floor{\lambda 2^n}$ posiciones de $x$. Es decir:
$$A^\lambda_{k,n}(x) = \frac{\#\{w: |w| = n  , |x[1...\floor{\lambda 2^n}]|_w = k\}}{2^n}$$
\pause
\begin{definition}
  Sea $\lambda$ un real mayor a cero. Decimos que la secuencia binaria $x$ es $\lambda$-supernormal si para todo entero no negativo $k$ sucede que
  $$\lim_{n\to\infty} A^\lambda_{k,n}(x) = \frac{e^{-\lambda}\lambda^k}{k!}$$

  Decimos que $x$ es supernormal si es $\lambda$-supernormal para todo $\lambda.$
\end{definition}
\end{frame}

\begin{frame}
  \frametitle{Un ejemplo para entender mejor}
  Veamos como ejemplo de juguete si la secuencia finita $x = 10011110$ es supernormal tomando $n=3$ y $\lambda = 1$.
  \pause
  Las palabras de tamaño 3 que ocurren en $x$ son:
  $$100, \; 001, \; 011,\; 111,\; 111,\; 110$$
  \pause
  Si contamos las cantidad de ocurrencias de cada palabra de tamaño 3 tenemos: 
  \begin{center}
    \begin{tabular}{|c | c|} 
    \hline
    Word & Count \\ [0.5ex] 
    \hline
    000 & 0 \\ 
    \hline
    001 & 1 \\ 
    \hline
    010 & 0 \\ 
    \hline
    011 & 1 \\ 
    \hline
    100 & 1 \\ 
    \hline
    101 & 0 \\ 
    \hline
    110 & 1 \\ 
    \hline
    111 & 2 \\ 
    \hline
   \end{tabular}
\end{center}
\end{frame}

\begin{frame}
  \frametitle{Un ejemplo para entender mejor}
  Ahora, veamos las cantidad, las frecuencias y el valor esperado para cada $k$ posible si $x$ fuera 1-supernormal.
  \pause
  \begin{center}
    \begin{tabular}{|c | c |  c| c |} 
    \hline
    $k$ & Count &  Frequency & Expected Frequency \\ [0.1ex] 
    \hline
    0 & 3 & $\frac{3}{8}$ & $e^{-1} $ \\ [0.5ex] 
    \hline
    1 & 4 &$\frac{1}{2}$ & $e^{-1} $ \\  [0.5ex] 
    \hline
    2 & 1 &$\frac{1}{8}$ & $\frac{e^{-1}}{2} $ \\  [0.5ex] 
    \hline
    3 & 0 & 0 & $\frac{e^{-1}}{3!} $ \\  [0.5ex] 
    \hline
    4 & 0 & 0 & $\frac{e^{-1}}{4!} $ \\ [0.5ex] 
    \hline
    5 & 0 & 0 & $\frac{e^{-1}}{5!} $ \\ [0.5ex] 
    \hline
    6 & 0 & 0  & $\frac{e^{-1}}{6!} $ \\ [0.5ex] 
    \hline
    7 & 0 & 0 & $\frac{e^{-1}}{7!} $ \\ [0.5ex] 
    \hline
    8 & 0 & 0 & $\frac{e^{-1}}{8!} $ \\  [0.5ex] 
    \hline
   \end{tabular}
\end{center}
\end{frame}

\begin{frame}
  \frametitle{El resultado de esta tesis}
  
  \begin{theorem}
    La noción de supernormalidad es más fuerte que la noción de normalidad.

    Es decir, los siguientes enunciados son ciertos:
    \begin{enumerate}
      \item Sea $x$ una secuencia infinita. Si $x$ es supernormal, entonces $x$ es normal. (supernormal $\Rightarrow$ normal )
      \item Sea $x$ una secuencia infinita. Si $x$ es normal, no necesariamente  $x$ es supernormal. (normal $\nRightarrow$ supernormal )
    \end{enumerate}
  \end{theorem}


\end{frame}

\end{document}


