\documentclass[10pt,mathserif]{beamer}%fleqn
\usepackage{amscd, amsthm,amssymb,latexsym}
\usepackage{xcolor}
\usepackage[utf8]{inputenc}
\usepackage[spanish]{babel}
\usepackage{subfig}

%\usepackage{beamerthemesplit}


\usepackage{url}
\mode<presentation>{
	\usecolortheme{}
	\useinnertheme{}
	\useoutertheme{}
}

%\usepackage{enumitem}
%\setenumerate{itemsep=0.005ex,topsep=3pt}
%\setitemize{itemsep=0.005ex,topsep=3pt}

\expandafter\def\expandafter\insertshorttitle\expandafter{%
  \insertshorttitle\hfill%
  \insertframenumber\,/\,\inserttotalframenumber}


\setbeamertemplate{headline}{}
\beamertemplatenavigationsymbolsempty


%\definecolor{magenta}{RGB}{250,0,150}
%\definecolor{black}{RGB}{250,0,150}

%\definecolor{mora}{RGB}{120,0,100}
%\definecolor{pink}{RGB}{255,100,160}

%\setbeamercolor{block title example}{bg=white,fg=red}
%\setbeamercolor{structure}{bg=white,fg=red}
%\setbeamercolor{title}{bg=white,fg=red}
%\setbeamercolor{normal text}{bg=white,fg=mora}
%\setbeamercolor{frametitle}{bg=white,fg=red}

\begin{document}

%\small

\binoppenalty=10000 
\relpenalty=10000
\hyphenpenalty=5000
\exhyphenpenalty=1000


\newtheorem{observation}{Observation}
\newtheorem{conjecture}{Conjecture}
\newtheorem{belief}{Belief}
\newtheorem{question}{Question}
\newtheorem{recall}{Recordemos}




\newcommand{\Z}{{\mathbb{Z}}}
\newcommand{\N}{{\mathbb{N}}}
\newcommand{\Q}{{\mathbb{Q}}}
\newcommand{\R}{{\mathbb{R}}}
\newcommand{\floor}[1]{\lfloor #1 \rfloor } 
\newcommand{\ceil}[1]{\lceil #1 \rceil }
\newcommand{\abs}[1]{\left| #1 \right|}
\newcommand{\card}{\mbox{\raisebox{.13em}{{$\scriptstyle \#$}}}}
\newcommand{\expa}[1]{\{#1\}}
\newcommand {\base}[2]{\langle{#1};{#2}\rangle}
\newcommand{\ybar}{{\overline{y}}}
\newcommand{\xbar}{{\overline{x}}}

\newcommand{\cf}{\text{\em cf}}
\newcommand{\eps}{\varepsilon}
\newcommand{\wh}[1]{\widehat{#1}}
\newcommand{\NN}{\mathbb{N}}
\newcommand{\RR}{\mathbb{R}}
\newcommand{\uno}{\mathbbm{1}}


\newcommand{\alocc}[2]{|\!|#1|\!|_{#2}}
\newcommand{\occ}[2]{|#1|_{#2}}

\languagepath{spanish}
\deftranslation[to=spanish]{Theorem}{Teorema}
\deftranslation[to=spanish]{theorem}{teorema}
\deftranslation[to=spanish]{Definition}{Definición}
\deftranslation[to=spanish]{definition}{definición}
\deftranslation[to=spanish]{Problem}{Problema}
\deftranslation[to=spanish]{problem}{problema}
\deftranslation[to=spanish]{Corollary}{Corolario}
\deftranslation[to=spanish]{corollary}{corolario}
\deftranslation[to=spanish]{Lemma}{Lema}
\deftranslation[to=spanish]{lemma}{lema}




%\title[Del azar con dos símbolos al azar con tres símbolos]{Del azar con dos símbolos al azar con tres símbolos}
%\author[Ariel Zylber]{Ariel Zylber}
%\institute{Universidad de Buenos Aires, Argentina}
%\date{\vspace*{-3cm}\footnotesize{Tesis de Licenciatura, Universidad de Buenos Aires, Noviembre 21, 2017}}

\title{{\normalsize Tesis de Licenciatura en Ciencias de la Computación}
\\\vspace*{2cm}
\mbox{\Large Números Muy Normales}}

\author{\large Lucas Puterman
\\\vspace*{1cm}}


\date{{\footnotesize
\hspace*{-6cm}
\begin{tabular}{l}
Directora: Ver\'onica Becher \\
Codirector: Olivier Carton\\
Departamento de Computaci\'on\\
Facultad de Ciencias Exactas y Naturales\\
 Universidad de Buenos Aires\\
19 de Noviembre, 2019
\end{tabular}
}}

\begin{frame}
\maketitle
\setcounter{framenumber}{0}
\thispagestyle{empty}
\end{frame}



\begin{frame}
\frametitle{Sobre secuencias aleatorias}

\begin{center}
  \includegraphics[scale=0.5]{imagenes/peso.jpg}
\end{center}

Supongamos que tiramos una moneda infinitas veces y anotamos un $1$ cada vez que sale cara y $0$ cada vez que sale ceca ¿Cuáles de estas secuencias es creíble que sea el resultado de este experimento?
\pause
\bigskip

\begin{itemize}
\item $11111111111111111111111111111111111\ldots$
\pause
\item $01010101010101010101010101010101010\ldots$
\pause
\item $10000110001010001110010010110011010\ldots$
\end{itemize}
\end{frame}


\begin{frame}
\frametitle{Secuencias normales}
Podemos pensar que en una secuencia aleatoria no hay ningún patrón de~$\ell$ símbolos que sea más frecuente que otro.\\
\pause
Nos gustaría que para un prefijo de una secuencia aleatoria suficientemente grande, 
la cantidad de ocurrencias de cada palabra de cierta longitud sea casi la misma.

\begin{definition}[{{\scriptsize  Borel, 1909}}]
  Dado un alfabeto $A$ y alguna secuencia infinita $u \in A^{\omega}$, 
  decimos que $u$ es \textit{normal} si para todos los tamaños de bloque ~$\ell$,
  sucede que todos los bloques posibles de tamaño ~$\ell$ ocurren con la misma frecuencia.
   \end{definition}
   \pause
   \begin{problem}[{{\scriptsize  Borel, 1909}}]
    Encontrar ejemplos naturales de secuencias normales. \\
    Decidir si la representación en base $b$ de $\pi$, $e$ ó $\sqrt{2}$ es normal.
    \end{problem}
\end{frame}



\begin{frame}
\frametitle{La secuencia de Champernowne}
  \begin{theorem}[{{\scriptsize  Champernowne, 1933}}]
  La secuencia
  $$1234567891011121314151617181920\ldots$$
  es normal 
  sobre 
  el alfabeto $A = \{0, 1, \ldots, 9\}$.
  \end{theorem}
\end{frame}

\begin{frame}
\frametitle{$champ$, La secuencia que usaremos}
  

  \begin{theorem}[{{\scriptsize  Champernowne, 1933}}]
    Sea $A$ un alfabeto. Llamamos  $X(n)$ a la concatenación de todas las palabras de longitud $n$ formadas por símbolos de $A$ en orden lexicográfico.
    
    La palabra infinita $X(1)X(2)\dots$ es normal en el alfabeto $A$
  \end{theorem}

  \medskip
  \pause
  En particular, nosotros vamos a usar el alfabeto  $A=\{0,1\}$ Entonces, por ejemplo:
  $$X(2) = 00 \: 01 \: 10 \: 11$$
  \medskip
  \pause
  Entonces, los primeros símbolos de la secuencia que llamamos $champ$ son:
  $$champ = 0 \: 1 \: 00 \: 01 \: 10 \: 11 \: 000 \: 001 \: 010 \: 011 \: 100 \: 101 \: 110 \: 111 \: 0000 \: 0001 \: \dots$$


\end{frame}

\begin{frame}
  \frametitle{Supernormalidad}
  Zeev Rudnick de la Universidad de Tel Aviv definió hace unos años la noción de supernomalidad con la que trabajamos en esta tesis.

  Benjamin Weiss de la Universidad Hebrea de Jerusalem dio el 16 de Junio de 2010 una conferencia en el Instituto de Altos Estudios de Princeton titulada “Random-like behavior in deterministic systems” donde describe la noción de supernormalidad, a la que llama “Poisson generic”.

  Lo poco que se conoce sobre esta noción no está publicado.

    \begin{figure}%
      \centering
      \subfloat{{\includegraphics[width=2cm]{imagenes/weiss.jpg} }}%
      \qquad
      \subfloat{{\includegraphics[width=2cm]{imagenes/rudnick.jpeg} }}%
      \caption{Benjamin Weiss y Zeev Rudnick}%
      % \label{fig:example}%
    \end{figure}
\end{frame}

\begin{frame}
  \frametitle{Que tal si contamos}
  $$x = 10011110\dots$$
  \\
  Contemos cuales son las palabras de tamaño 3 que aparecen en los primeros 8 símbolos de $x$.
  \pause
  $$100\pause, \; 001\pause, \ \; 011\pause,\; 111\pause,\; 111\pause,\; 110$$
  \pause
  Si contamos las cantidad de ocurrencias de cada palabra de tamaño 3 tenemos: 
  \begin{columns}
    \begin{column}{0.5\textwidth}
      \begin{center}
      \begin{tabular}{|c | c|} 
    \hline
    Palabra & Cantidad \\ [0.5ex] 
    \hline
    000 & 0 \\ 
    \hline
    001 & 1 \\ 
    \hline
    010 & 0 \\ 
    \hline
    011 & 1 \\ 
    \hline
    100 & 1 \\ 
    \hline
    101 & 0 \\ 
    \hline
    110 & 1 \\ 
    \hline
    111 & 2 \\ 
    \hline
   \end{tabular}
  \end{center}
    \end{column}
    \begin{column}{0.5\textwidth}
      \pause
      \begin{center}
      \begin{tabular}{|c | c |  c|} 
        \hline
        $k$ & Cant &  Frec \\ [0.5ex] 
        \hline
        0 & 3 & $3/8$ \\ 
        \hline
        1 & 4 &$1/2$\\ 
        \hline
        2 & 1 &$1/8$  \\  
        \hline
        3 & 0 & 0 \\ 
        \hline
        4 & 0 & 0 \\
        \hline
        5 & 0 & 0  \\  
        \hline
        6 & 0 & 0  \\ 
        \hline
        7 & 0 & 0 \\
        \hline
        8 & 0 & 0 \\  
        \hline
       \end{tabular}
      \end{center}
    \end{column}
    \end{columns}
\end{frame}

% \begin{frame}
%   \frametitle{Que tal si contamos}
% %   Lo que le vamos a pedir a estas frecuencias es que los valores no sean cualquiera.Queremos que sigan una distribución de Poisson, en este caso de parámetero 1.
  
% %   Es decir, buscamos que la cantidad de palabras que aparecen $k$ veces sobre la cantidad de palabras totales sea:

% %   $$\frac{1^k e^{-1}}{k!}$$
% %   \pause
% %   La distribución Poisson es una ditribución de probabilidad discreta que expresa la probabilidad de que una cantidad de eventos ocurran en un intervalo fijo de tiempo si estos ocurren con una frecuencia media conocida e independientemente.
% %   En este caso la frecuencia media es 1. Intuitivamente, le estamos dando a cada una de las palabras una oportunidad de aparecer.

% %   Si en vez de considerar los primeros 8 símbolos de $x$ consideramos los primeros 16, entonces esta frecuencia media sería 2.
% %   Llamamos $\lambda$ a este valor que regula la cantidad de símbolos de $x$ que vamos a leer.

%   Lo que le vamos a pedir a estas frecuencias es que los valores no sean cualquiera.

% Buscamos que la proporcio de palabras que aparecen k veces, respecto de la cantidad total sea:

% $$e^{-1} / k!$$

% Esto es un caso especial de la distribucion de Poisson, que expresa la probabilidad de que una cantidad de eventos ocurran en un intervalo de tiempo si conocemos la frecuencia media de ocurrencia (que en este caso particular sería 1). Intuitivamente, le estamos dando a cada una de las palabras una oportunidad de aparecer .

% Si quisieramos considerar otra frecuencia de aparición, deberíamos cambiar la cantidad de símbolos que leemos: si fuera 2 en vez de 1, 
% considerariamos los primeros 16 símbolos (el doble de 8). Llamamos $\lambda$ a este valor, que va regular la cantidad de símbolos que vamos a leer.
  
  
% \end{frame}

\begin{frame}
\frametitle{Supernormalidad}
Sea $x$ una secuencia binaria. Sea $A^\lambda_{k,n}(x)$  la frecuencia de ocurrencia de las palabras de longitud $n$ que ocurren exactamente $k$ veces comenzando en las primeras $\floor{\lambda 2^n}$ posiciones de $x$. Es decir:
% \begin{align*}

% A^\lambda_{k,n}(x) =& \frac{\#\{w: |w| = n  , \text{cant. de ocurrencias de $w$ en $x[1\dots\floor{\lambda 2^n}]$ } = k\}}{2^n}\\
% \pause
% \bigskip
% A^\lambda_{k,n}(x) =& \frac{\#\{w: |w| = n  , |x[1...\floor{\lambda 2^n}]|_w = k\}}{2^n}\\
% \pause

% \end{align*}

$$A^\lambda_{k,n}(x) = \frac{\#\{w: |w| = n  , \text{cant. de ocurrencias de $w$ en $x[1\dots\floor{\lambda 2^n}]$ } = k\}}{2^n}$$
\pause
\bigskip
$$A^\lambda_{k,n}(x) = \frac{\#\{w: |w| = n  , |x[1...\floor{\lambda 2^n}]|_w = k\}}{2^n}$$
\pause
\begin{definition}
  Sea $\lambda$ un real mayor a cero. Decimos que la secuencia binaria $x$ es $\lambda$-supernormal si para todo entero no negativo $k$ sucede que
  
  $$\lim_{n\to\infty} A^\lambda_{k,n}(x) = \frac{e^{-\lambda}\lambda^k}{k!}$$

  Decimos que $x$ es supernormal si es $\lambda$-supernormal para todo $\lambda.$
\end{definition}
\end{frame}



\begin{frame}
  \frametitle{El resultado de esta tesis}
  \pause
  \begin{theorem}
    La noción de supernormalidad es más fuerte que la noción de normalidad.

    Es decir, los siguientes enunciados son ciertos:
    \begin{enumerate}
      \item Sea $x$ una secuencia infinita. Si $x$ es normal, no necesariamente  $x$ es supernormal. (normal $\nRightarrow$ supernormal )
      \item Sea $x$ una secuencia infinita. Si $x$ es supernormal, entonces $x$ es normal. (supernormal $\Rightarrow$ normal )
    \end{enumerate}
  \end{theorem}
  \pause
  \begin{figure}[h]
    \includegraphics[width=0.4\textwidth]{imagenes/meme.png}
    \centering
    \label{fig:meme}
\end{figure}
\end{frame}

\begin{frame}
  \frametitle{¿Es $champ$ supernormal?}
  
  La forma más simple de ver que una secuencia normal no es supernormal es encontrar un ejemplo, y que mejor ejemplo que $champ$.

  \begin{recall}
    $X(n)$ es la concatenación de todas las palabras de longitud $n$ sobre el alfabeto $A=\{0,1\}$ en order lexicográfico.\\
    
    Llamamos $champ$ a la concatenación de  $X(n)$ para $n = 1,2,\dots$

    $$champ = 0 \: 1 \: 00 \: 01 \: 10 \: 11 \: 000 \: 001 \: 010 \: 011 \: 100 \: 101 \: 110 \: 111 \: 0000 \: 0001 \: \dots$$

  \end{recall}

  \pause
  ¿Qué sucede cuando contamos la cantidad de palabras de tamaño $n$ en los primeros $2^n$ símbolos de $champ$?
  
\end{frame}


\begin{frame}
  \frametitle{¿Es $champ$ supernormal?}
  \begin{figure}[h]
    \includegraphics[width=0.75\textwidth]{imagenes/champ-16-freq.png}
    \centering
    \caption{Frecuencias observadas y esperadas en $champ$ for $n = 16$ y $\lambda = 1$.}
    \label{fig-16-freq}
\end{figure}
\end{frame}

\begin{frame}
  \frametitle{¿Es $champ$ supernormal?}
  \begin{figure}[h]
    \includegraphics[width=0.75\textwidth]{imagenes/champ-22-freq.png}
    \centering
    \caption{Frecuencias observadas y esperadas en $champ$ for $n = 22$ y $\lambda = 1$.}
    \label{fig:champ-22-freq}
\end{figure}
\end{frame}

\begin{frame}
  \frametitle{Normal $\nRightarrow$ Supernormal - Idea de la demostración}
  \begin{itemize}
    \item Sabemos que $champ = X(1)X(2)X(3)\dots$
    \pause 
    \item Vamos a buscar el $k$ más grande tal que $X(k)$ este completamente contenido en $champ[1\dots 2^n]$ y contar cuales son las palabras que aparecen adentro.
    \pause 
    \item Para esto buscamos una forma de elegir algunos $n$s que tengan una pinta que nos sirva.
    \pause
    \item Encontramos que si tomamos $n = d + k + 1$ con $k = 2^d$, sucede que $X(k)$ está completamente contenido en $champ[1\dots 2^n]$ y además ocupa la mitad del espacio.
    \pause 
    \item Nos fijamos cuales son las palabras de longitud $d + k + 1$ que aparecen adentro de $X(k)$.
    \pause
    \item Conociendo las palabras que aparecen, vemos que las son más de lo necesario para que $champ$ sea supernormal.
  \end{itemize}
\end{frame}

\begin{frame}
  \frametitle{Que hay adentro de $X(k)$}
  Recordemos que definimos $k = 2^d$.
  Veamos como son las palabras de tamaño $d + k + 1$ que suceden adentro de $X(k)$. Tomemos a modo de ejemplo $k = 8$ y $d = 3$.
  \pause
  $$00000000 \qquad 00000001 \qquad 00000010 \qquad 00000011 \qquad 00000100$$
  \phantom{\parbox{\linewidth}{%
    \begin{itemize}
    \item \textcolor{red}{\underline{Caso 1:}}
    $$x = u_1 u_2 \dots u_k \quad v_1 v_2 \dots v_{d} v_{d + 1}$$
    \item \textcolor{blue}{\underline{Caso 2:}}
    $$ x = u_{k-d-1} \dots u_k \quad v_1 v_2 \dots v_k$$
    \item \textcolor{magenta}{\underline{Caso 3:}}
    $$x = u_{n+1} u_{n+2} \dots u_k \quad  v_1 v_2 \dots v_{d+n+1} $$
    con $n \in \{1,2,\dots ,k - d - 2\}$.
    \item \textcolor{orange}{\underline{Caso 4:}}
    $$ x = u_{k-d-1+n} u_{k-d+n} \dots u_k \quad v_1 v_2 \dots v_k \quad w_1 w_2 \dots w_{d+1-n}$$
    con $n \in \{1, 2, \dots , d\}$
  \end{itemize}}}\par
\end{frame} 

\begin{frame}
  \frametitle{Que hay adentro de $X(k)$}
  Recordemos que definimos $k = 2^d$.
  Veamos como son las palabras de tamaño $d + k + 1$ que suceden adentro de $X(k)$. Tomemos a modo de ejemplo $k = 8$ y $d = 3$.

  $$\textcolor{red}{(00000000 \qquad 0000)} 0001 \qquad 00000010 \qquad 00000011 \qquad 00000100$$

    \begin{itemize}
    \item \textcolor{red}{\underline{Caso 1:}}
    $$x = u_1 u_2 \dots u_k \quad v_1 v_2 \dots v_{d} v_{d + 1}$$
    \phantom{\parbox{\linewidth}{%
    \item \textcolor{blue}{\underline{Caso 2:}}
    $$ x = u_{k-d-1} \dots u_k \quad v_1 v_2 \dots v_k$$
    \item \textcolor{magenta}{\underline{Caso 3:}}
    $$x = u_{n+1} u_{n+2} \dots u_k \quad  v_1 v_2 \dots v_{d+n+1} $$
    con $n \in \{1,2,\dots ,k - d - 2\}$.
    \item \textcolor{cyan}{\underline{Caso 4:}}
    $$ x = u_{k-d-1+n} u_{k-d+n} \dots u_k \quad v_1 v_2 \dots v_k \quad w_1 w_2 \dots w_{d+1-n}$$
    con $n \in \{1, 2, \dots , d\}$
    }}\par
  \end{itemize}
\end{frame} 

\begin{frame}
  \frametitle{Que hay adentro de $X(k)$}
  Recordemos que definimos $k = 2^d$.
  Veamos como son las palabras de tamaño $d + k + 1$ que suceden adentro de $X(k)$. Tomemos a modo de ejemplo $k = 8$ y $d = 3$.

  $$0\textcolor{magenta}{(0000000 \qquad 00000)} 001 \qquad 00000010 \qquad 00000011 \qquad 00000100$$

    \begin{itemize}
    \item \textcolor{red}{\underline{Caso 1:}}
    $$x = u_1 u_2 \dots u_k \quad v_1 v_2 \dots v_{d} v_{d + 1}$$  
    \phantom{\parbox{\linewidth}{%
    \item \textcolor{blue}{\underline{Caso 2:}}
    $$ x = u_{k-d-1} \dots u_k \quad v_1 v_2 \dots v_k$$
    }}\par
    \item \textcolor{magenta}{\underline{Caso 3:}}
    $$x = u_{n+1} u_{n+2} \dots u_k \quad  v_1 v_2 \dots v_{d+n+1} $$
    con $n \in \{1,2,\dots ,k - d - 2\}$.
    \phantom{\parbox{\linewidth}{%
    \item \textcolor{cyan}{\underline{Caso 4:}}
    $$ x = u_{k-d-1+n} u_{k-d+n} \dots u_k \quad v_1 v_2 \dots v_k \quad w_1 w_2 \dots w_{d+1-n}$$
    con $n \in \{1, 2, \dots , d\}$
    }}\par
  \end{itemize}
\end{frame}

\begin{frame}
  \frametitle{Que hay adentro de $X(k)$}
  Recordemos que definimos $k = 2^d$.
  Veamos como son las palabras de tamaño $d + k + 1$ que suceden adentro de $X(k)$. Tomemos a modo de ejemplo $k = 8$ y $d = 3$.

  $$00\textcolor{magenta}{(000000 \qquad 000000)} 01 \qquad 00000010 \qquad 00000011 \qquad 00000100$$

    \begin{itemize}
    \item \textcolor{red}{\underline{Caso 1:}}
    $$x = u_1 u_2 \dots u_k \quad v_1 v_2 \dots v_{d} v_{d + 1}$$  
    \phantom{\parbox{\linewidth}{%
    \item \textcolor{blue}{\underline{Caso 2:}}
    $$ x = u_{k-d-1} \dots u_k \quad v_1 v_2 \dots v_k$$
    }}\par
    \item \textcolor{magenta}{\underline{Caso 3:}}
    $$x = u_{n+1} u_{n+2} \dots u_k \quad  v_1 v_2 \dots v_{d+n+1} $$
    con $n \in \{1,2,\dots ,k - d - 2\}$.
    \phantom{\parbox{\linewidth}{%
    \item \textcolor{cyan}{\underline{Caso 4:}}
    $$ x = u_{k-d-1+n} u_{k-d+n} \dots u_k \quad v_1 v_2 \dots v_k \quad w_1 w_2 \dots w_{d+1-n}$$
    con $n \in \{1, 2, \dots , d\}$
    }}\par
  \end{itemize}
\end{frame}

\begin{frame}
  \frametitle{Que hay adentro de $X(k)$}
  Recordemos que definimos $k = 2^d$.
  Veamos como son las palabras de tamaño $d + k + 1$ que suceden adentro de $X(k)$. Tomemos a modo de ejemplo $k = 8$ y $d = 3$.

  $$000\textcolor{magenta}{(00000 \qquad 0000000)} 1 \qquad 00000010 \qquad 00000011 \qquad 00000100$$

    \begin{itemize}
    \item \textcolor{red}{\underline{Caso 1:}}
    $$x = u_1 u_2 \dots u_k \quad v_1 v_2 \dots v_{d} v_{d + 1}$$  
    \phantom{\parbox{\linewidth}{%
    \item \textcolor{blue}{\underline{Caso 2:}}
    $$ x = u_{k-d-1} \dots u_k \quad v_1 v_2 \dots v_k$$
    }}\par
    \item \textcolor{magenta}{\underline{Caso 3:}}
    $$x = u_{n+1} u_{n+2} \dots u_k \quad  v_1 v_2 \dots v_{d+n+1} $$
    con $n \in \{1,2,\dots ,k - d - 2\}$.
    \phantom{\parbox{\linewidth}{%
    \item \textcolor{cyan}{\underline{Caso 4:}}
    $$ x = u_{k-d-1+n} u_{k-d+n} \dots u_k \quad v_1 v_2 \dots v_k \quad w_1 w_2 \dots w_{d+1-n}$$
    con $n \in \{1, 2, \dots , d\}$
    }}\par
  \end{itemize}
\end{frame}

\begin{frame}
  \frametitle{Que hay adentro de $X(k)$}
  Recordemos que definimos $k = 2^d$.
  Veamos como son las palabras de tamaño $d + k + 1$ que suceden adentro de $X(k)$. Tomemos a modo de ejemplo $k = 8$ y $d = 3$.

  $$0000\textcolor{blue}{(0000 \qquad 00000001)}  \qquad 00000010 \qquad 00000011 \qquad 00000100$$

    \begin{itemize}
    \item \textcolor{red}{\underline{Caso 1:}}
    $$x = u_1 u_2 \dots u_k \quad v_1 v_2 \dots v_{d} v_{d + 1}$$  
    \item \textcolor{blue}{\underline{Caso 2:}}
    $$ x = u_{k-d-1} \dots u_k \quad v_1 v_2 \dots v_k$$
    \item \textcolor{magenta}{\underline{Caso 3:}}
    $$x = u_{n+1} u_{n+2} \dots u_k \quad  v_1 v_2 \dots v_{d+n+1} $$
    con $n \in \{1,2,\dots ,k - d - 2\}$.
    \phantom{\parbox{\linewidth}{%
    \item \textcolor{cyan}{\underline{Caso 4:}}
    $$ x = u_{k-d-1+n} u_{k-d+n} \dots u_k \quad v_1 v_2 \dots v_k \quad w_1 w_2 \dots w_{d+1-n}$$
    con $n \in \{1, 2, \dots , d\}$
    }}\par
  \end{itemize}
\end{frame}

\begin{frame}
  \frametitle{Que hay adentro de $X(k)$}
  Recordemos que definimos $k = 2^d$.
  Veamos como son las palabras de tamaño $d + k + 1$ que suceden adentro de $X(k)$. Tomemos a modo de ejemplo $k = 8$ y $d = 3$.

  $$00000\textcolor{cyan}{(000 \qquad 00000001 \qquad 0)}  0000010 \qquad 00000011 \qquad 00000100$$

    \begin{itemize}
    \item \textcolor{red}{\underline{Caso 1:}}
    $$x = u_1 u_2 \dots u_k \quad v_1 v_2 \dots v_{d} v_{d + 1}$$  
    \item \textcolor{blue}{\underline{Caso 2:}}
    $$ x = u_{k-d-1} \dots u_k \quad v_1 v_2 \dots v_k$$
    \item \textcolor{magenta}{\underline{Caso 3:}}
    $$x = u_{n+1} u_{n+2} \dots u_k \quad  v_1 v_2 \dots v_{d+n+1} $$
    con $n \in \{1,2,\dots ,k - d - 2\}$.
    \item \textcolor{cyan}{\underline{Caso 4:}}
    $$ x = u_{k-d-1+n} u_{k-d+n} \dots u_k \quad v_1 v_2 \dots v_k \quad w_1 w_2 \dots w_{d+1-n}$$
    con $n \in \{1, 2, \dots , d\}$
  \end{itemize}
\end{frame}

\begin{frame}
  \frametitle{Que hay adentro de $X(k)$}
  Recordemos que definimos $k = 2^d$.
  Veamos como son las palabras de tamaño $d + k + 1$ que suceden adentro de $X(k)$. Tomemos a modo de ejemplo $k = 8$ y $d = 3$.

  $$000000\textcolor{cyan}{(00 \qquad 00000001 \qquad 00)}  000010 \qquad 00000011 \qquad 00000100$$

    \begin{itemize}
    \item \textcolor{red}{\underline{Caso 1:}}
    $$x = u_1 u_2 \dots u_k \quad v_1 v_2 \dots v_{d} v_{d + 1}$$  
    \item \textcolor{blue}{\underline{Caso 2:}}
    $$ x = u_{k-d-1} \dots u_k \quad v_1 v_2 \dots v_k$$
    \item \textcolor{magenta}{\underline{Caso 3:}}
    $$x = u_{n+1} u_{n+2} \dots u_k \quad  v_1 v_2 \dots v_{d+n+1} $$
    con $n \in \{1,2,\dots ,k - d - 2\}$.
    \item \textcolor{cyan}{\underline{Caso 4:}}
    $$ x = u_{k-d-1+n} u_{k-d+n} \dots u_k \quad v_1 v_2 \dots v_k \quad w_1 w_2 \dots w_{d+1-n}$$
    con $n \in \{1, 2, \dots , d\}$
  \end{itemize}
\end{frame}

\begin{frame}
  \frametitle{Que hay adentro de $X(k)$}
  Recordemos que definimos $k = 2^d$.
  Veamos como son las palabras de tamaño $d + k + 1$ que suceden adentro de $X(k)$. Tomemos a modo de ejemplo $k = 8$ y $d = 3$.

  $$0000000\textcolor{cyan}{(0 \qquad 00000001 \qquad 000)}  00010 \qquad 00000011 \qquad 00000100$$

    \begin{itemize}
    \item \textcolor{red}{\underline{Caso 1:}}
    $$x = u_1 u_2 \dots u_k \quad v_1 v_2 \dots v_{d} v_{d + 1}$$  
    \item \textcolor{blue}{\underline{Caso 2:}}
    $$ x = u_{k-d-1} \dots u_k \quad v_1 v_2 \dots v_k$$
    \item \textcolor{magenta}{\underline{Caso 3:}}
    $$x = u_{n+1} u_{n+2} \dots u_k \quad  v_1 v_2 \dots v_{d+n+1} $$
    con $n \in \{1,2,\dots ,k - d - 2\}$.
    \item \textcolor{cyan}{\underline{Caso 4:}}
    $$ x = u_{k-d-1+n} u_{k-d+n} \dots u_k \quad v_1 v_2 \dots v_k \quad w_1 w_2 \dots w_{d+1-n}$$
    con $n \in \{1, 2, \dots , d\}$
  \end{itemize}
\end{frame}

\begin{frame}
  \frametitle{Que hay adentro de $X(k)$}
  Recordemos que definimos $k = 2^d$.
  Veamos como son las palabras de tamaño $d + k + 1$ que suceden adentro de $X(k)$. Tomemos a modo de ejemplo $k = 8$ y $d = 3$.

  $$00000000 \qquad \textcolor{red}{(00000001 \qquad 0000)} 0010 \qquad 00000011 \qquad 00000100 $$

    \begin{itemize}
    \item \textcolor{red}{\underline{Caso 1:}}
    $$x = u_1 u_2 \dots u_k \quad v_1 v_2 \dots v_{d} v_{d + 1}$$
    \item \textcolor{blue}{\underline{Caso 2:}}
    $$ x = u_{k-d-1} \dots u_k \quad v_1 v_2 \dots v_k$$
    \item \textcolor{magenta}{\underline{Caso 3:}}
    $$x = u_{n+1} u_{n+2} \dots u_k \quad  v_1 v_2 \dots v_{d+n+1} $$
    con $n \in \{1,2,\dots ,k - d - 2\}$.
    \item \textcolor{cyan}{\underline{Caso 4:}}
    $$ x = u_{k-d-1+n} u_{k-d+n} \dots u_k \quad v_1 v_2 \dots v_k \quad w_1 w_2 \dots w_{d+1-n}$$
    con $n \in \{1, 2, \dots , d\}$
  \end{itemize}
\end{frame} 

\begin{frame}
  \frametitle{Que hay adentro de $X(k)$}
  Recordemos que definimos $k = 2^d$.
  Veamos como son las palabras de tamaño $d + k + 1$ que suceden adentro de $X(k)$. Tomemos a modo de ejemplo $k = 8$ y $d = 3$.

  $$11111011 \qquad 11111100 \qquad 11111101 \qquad 11111\textcolor{green}{(110 \qquad 11111111} \qquad 0\dots $$

    \begin{itemize}
    \item \textcolor{red}{\underline{Caso 1:}}
    $$x = u_1 u_2 \dots u_k \quad v_1 v_2 \dots v_{d} v_{d + 1}$$
    \item \textcolor{blue}{\underline{Caso 2:}}
    $$ x = u_{k-d-1} \dots u_k \quad v_1 v_2 \dots v_k$$
    \item \textcolor{magenta}{\underline{Caso 3:}}
    $$x = u_{n+1} u_{n+2} \dots u_k \quad  v_1 v_2 \dots v_{d+n+1} $$
    con $n \in \{1,2,\dots ,k - d - 2\}$.
    \item \textcolor{cyan}{\underline{Caso 4:}}
    $$ x = u_{k-d-1+n} u_{k-d+n} \dots u_k \quad v_1 v_2 \dots v_k \quad w_1 w_2 \dots w_{d+1-n}$$
    con $n \in \{1, 2, \dots , d\}$
    \item \textcolor{green}{\underline{Caso 5:}}
    Si $x$ comienza al final de $X(k)$ o cerca, se pueden llegar a necesitar palabras fuera de $X(k)$ para completar $d+k+1$.
  \end{itemize}
\end{frame} 

% \begin{frame}
%   \frametitle{La función $next(w)$}

%   Definimos la siguiente función que nos va a ser útil.

%   \begin{definition}
%     La función $next(w) : A^n \rightarrow A^n$ se define de la siguiente manera. Si $w$ es la palabra de $n$  1s, $next(w)$ es la palabra de $n$ 0s. Si no, $next(w)$ es la palabra siguiente en orden lexicográfico.
%   \end{definition}
  
%   Por ejemplo,
%   $$next(0000) = 0001$$
%   $$next(0001) = 0010$$
%   $$\vdots$$
%   $$next(1110) = 1111$$
%   $$next(1111) = 0000$$

% \end{frame} 

\begin{frame}
  \frametitle{Análisis por caso - Caso 1}
    $$x = u_1 u_2 \dots u_k \quad v_1 v_2 \dots v_{d} v_{d + 1}$$

  Este caso representa a la ocurrencia alineada de $u$ de tamaño $k$ seguido por los primeros $d+1$ símbolos de $v$

  \begin{columns}
    \begin{column}{0.5\textwidth}
      $$( 00000000 \qquad 0000 ) \; 0001$$
      $$( 00000001 \qquad 0000 ) \; 0010$$
      $$( 00000010 \qquad 0000 ) \; 0011$$
      $$\vdots$$
      $$( 00001110 \qquad 0000 ) \; 1111$$
      $$( 00001111 \qquad 0001 ) \; 0000$$
      $$( 00010000 \qquad 0001 ) \; 0001$$
      $$\vdots$$
      $$( 11111110 \qquad 1111 ) \; 1111$$
    \end{column}
    \begin{column}{0.5\textwidth}  %%<--- here
        \begin{center}
        %  \includegraphics[width=0.5\textwidth]{image1.jpg}
         \end{center}
    \end{column}
    \end{columns}



\end{frame} 

\begin{frame}
  \frametitle{Análisis por caso - Caso 1}
    $$x = u_1 u_2 \dots u_k \quad v_1 v_2 \dots v_{d} v_{d + 1}$$

  Este caso representa a la ocurrencia alineada de $u$ de tamaño $k$ seguido por los primeros $d+1$ símbolos de $v$
  
  \begin{columns}
    \begin{column}{0.5\textwidth}
      $$( \textcolor{red}{0000}0000 \qquad \textcolor{red}{0000} ) \; 0001$$
      $$( \textcolor{red}{0000}0001 \qquad \textcolor{red}{0000} ) \; 0010$$
      $$( \textcolor{red}{0000}0010 \qquad \textcolor{red}{0000} ) \; 0011$$
      $$\vdots$$
      $$( \textcolor{red}{0000}1110 \qquad \textcolor{red}{0000} ) \; 1111$$
      $$( \textcolor{blue}{0000}\textcolor{magenta}{1111} \qquad \textcolor{cyan}{0001} ) \; 0000$$
      $$( 00010000 \qquad 0001 ) \; 0001$$
      $$\vdots$$
      $$( 11111110 \qquad 1111 ) \; 1111$$
    \end{column}
    \begin{column}{0.5\textwidth}  %%<--- here
          \pause
          $$\underbrace{\quad \textcolor{red}{A} \quad }_{d +1} \qquad \underbrace{\quad B \quad }_{k - d - 1}  \qquad \textcolor{red}{A}$$
          $$\underbrace{\quad \textcolor{blue}{A} \quad }_{d +1} \qquad \underbrace{ \textcolor{magenta}{11 \dots 1  }}_{k - d - 1}  \qquad \textcolor{cyan}{next(A)} $$
          \pause
          Del primer esquema tenemos:
          $$2^{d + 1}  (2^{k - d - 1} - 1) =  2^k - \frac{1}{2\cdot2^{d} }$$
        
          Mientras que del segundo:
          $$2^{d + 1}   - 1$$

          Juntos son menos que $2^k -  2\cdot 2^d$
    \end{column}
    \end{columns}
\end{frame} 

\begin{frame}
  \frametitle{Análisis por caso - Caso 2}
  $$ x = u_{k-d-1} \dots u_k \quad v_1 v_2 \dots v_k$$

  Este caso representa a la ocurrencia alineada de $v$ de tamaño $k$ precedida por los últimos $d+1$ símbolos de $u$
  
  \begin{columns}
    \begin{column}{0.5\textwidth}
      $$0000 \; (0000 \qquad 00000001)$$
      $$0000 \; (\; 0001 \qquad 00000010)$$
      $$\vdots$$
      $$0000 \; (1111 \qquad 10000000)$$
      $$1000 \; (0000 \qquad 10000001)$$
      $$\vdots$$
      $$1111 \; (1110 \qquad 11111111)$$
    \end{column}
    \begin{column}{0.5\textwidth}  %%<--- here
          
    \end{column}
    \end{columns}
\end{frame} 

\begin{frame}
  \frametitle{Análisis por caso - Caso 2}
  $$ x = u_{k-d-1} \dots u_k \quad v_1 v_2 \dots v_k$$

  Este caso representa a la ocurrencia alineada de $v$ de tamaño $k$ precedida por los últimos $d+1$ símbolos de $u$
  
  \begin{columns}
    \begin{column}{0.5\textwidth}
      $$0000 \; (\textcolor{red}{0000} \qquad 0000\textcolor{blue}{0001})$$
      $$0000 \; (\; \textcolor{red}{0001} \qquad 0000\textcolor{blue}{0010})$$
      $$\vdots$$
      $$0000 \; (\textcolor{red}{1111} \qquad 1000\textcolor{blue}{0000})$$
      $$1000 \; (\textcolor{red}{0000} \qquad 1000\textcolor{blue}{0001})$$
      $$\vdots$$
      $$1111 \; (\textcolor{red}{1110} \qquad 1111\textcolor{blue}{1111})$$
    \end{column}
    \begin{column}{0.5\textwidth}  %%<--- here
          \pause
          $$\underbrace{\quad \textcolor{red}{A} \quad }_{d +1} \qquad \underbrace{\quad B \quad }_{k - d - 1}  \qquad \textcolor{blue}{next(A)}$$
          \pause
          Este esquema nos da:
          $$(2^{d + 1} 2^{k-d})-1=  2 \cdot 2^k - 1$$
          palabras distintas.

          Que es menos que  $2 \cdot 2^k$ palabras distintas.
    \end{column}
    \end{columns}
\end{frame} 

\begin{frame}
  \frametitle{Análisis por caso - Caso 3}
  $$x = u_{n+1} u_{n+2} \dots u_k \quad  v_1 v_2 \dots v_{d+n+1}   \qquad n \in \{1,2,\dots ,k - d - 2\}  $$
  Este caso representa cuando los $k + d + 1$ símbolos son tomados de dos palabras $u$ y $v$ de longitud $k$ y ninguna de las dos está completas.
  Ejemplo tomando $n = 3$, $k = 8$ y $d = 3$.
  \begin{columns}
    \begin{column}{0.5\textwidth}
        $$000\; (0000\; 0 \qquad 000 \;0000 ) \;1$$
        $$000\; (0000\; 1 \qquad 000 \;0001 ) \;0$$
        $$000\; (0001\; 0 \qquad 000 \;0001 ) \;1$$
        $$000\; (0001\; 1 \qquad 000 \;0010 ) \;0$$
        $$\vdots$$
        $$111\; (1110\; 1 \qquad 111 \;1111 ) \;0$$
        $$111\; (1111\; 0 \qquad 111 \;1111 ) \;1$$
    \end{column}
    \begin{column}{0.5\textwidth}  %%<--- here

    \end{column}
    \end{columns}
\end{frame}

\begin{frame}
  \frametitle{Análisis por caso - Caso 3}
  $$x = u_{n+1} u_{n+2} \dots u_k \quad  v_1 v_2 \dots v_{d+n+1}   \qquad n \in \{1,2,\dots ,k - d - 2\}  $$
  Este caso representa cuando los $k + d + 1$ símbolos son tomados de dos palabras $u$ y $v$ de longitud $k$ y ninguna de las dos está completas.
  Ejemplo tomando \textbf{$n = 1$}, $k = 8$ y $d = 3$.
  \begin{columns}
    \begin{column}{0.5\textwidth}
      \begin{small}

      $$0\; (0000\; 000 \qquad 0 \;0000 ) \;001$$
      $$0\; (0000 \;001 \qquad 0 \;0000 ) \;010$$
      $$\vdots$$
      $$0\; (0001 \;110 \qquad 0 \;0001 ) \;111$$
      $$0\; (0001 \;111 \qquad 0 \;0010 ) \;000$$
      $$0\; (0010 \;000 \qquad 0 \;0010 ) \;001$$
      $$\vdots$$
      $$1\; (1111 \;101 \qquad 1 \;1111 ) \;110$$
      $$0\; (1111 \;110 \qquad 1 \;1111 ) \;111$$
      \end{small}
    \end{column}
    \begin{column}{0.5\textwidth}  %%<--- here
          
    \end{column}
    \end{columns}
\end{frame}

\begin{frame}
  \frametitle{Análisis por caso - Caso 3}
  $$x = u_{n+1} u_{n+2} \dots u_k \quad  v_1 v_2 \dots v_{d+n+1}   \qquad n \in \{1,2,\dots ,k - d - 2\}  $$
  Este caso representa cuando los $k + d + 1$ símbolos son tomados de dos palabras $u$ y $v$ de longitud $k$ y ninguna de las dos está completas.
  Ejemplo tomando $n = 1$, $k = 8$ y $d = 3$.
  \begin{columns}
    \begin{column}{0.5\textwidth}
      \begin{small}

      $$0\; (\textcolor{red}{0000}\; 000 \qquad 0 \;\textcolor{red}{0000} ) \;001$$
      $$0\; (\textcolor{red}{0000} \;001 \qquad 0 \;\textcolor{red}{0000} ) \;010$$
      $$\vdots$$
      $$0\; (\textcolor{red}{0001} \;110 \qquad 0 \;\textcolor{red}{0001} ) \;111$$
      $$0\; (\textcolor{red}{0001} \;\textcolor{blue}{111} \qquad 0 \;\textcolor{cyan}{0010} ) \;000$$
      $$0\; (\textcolor{red}{0010} \;000 \qquad 0 \;\textcolor{red}{0010} ) \;001$$
      $$\vdots$$
      $$1\; (\textcolor{red}{1111} \;101 \qquad 1 \;\textcolor{red}{1111} ) \;110$$
      $$0\; (\textcolor{red}{1111} \;110 \qquad 1 \;\textcolor{red}{1111} ) \;111$$
      \end{small}
    \end{column}
    \begin{column}{0.5\textwidth}  %%<--- here
      \pause
      $$\underbrace{\quad \textcolor{red}{A} \quad }_{d +1} \qquad \underbrace{\quad B \quad }_{k - d - 1}  \qquad \textcolor{red}{A}$$
      $$\underbrace{\quad \textcolor{red}{A} \quad }_{d +1} \qquad \underbrace{\; \textcolor{blue}{11\dots1}C \; }_{k - d - 1}  \qquad \textcolor{cyan}{next(A)}$$
      \pause
      Pero el primer esquema ya lo contamos en el caso 1 y el segundo esquema es un caso particular de uno de los esquemas del caso 2.
      
      Por lo que el caso 3 no nos arroja nuevas palabras.
    \end{column}
    \end{columns}
\end{frame}

\begin{frame}
  \frametitle{Análisis por caso - Caso 4}
  $$ x = u_{k-d-1+n} u_{k-d+n} \dots u_k \quad v_1 v_2 \dots v_k \quad w_1 w_2 \dots w_{d+1-n} $$
  con $n \in \{1, 2, \dots , d\}$

  Este caso representa cuando los $k + d + 1$ símbolos son tomados de tres palabras consecutivas $u$, $v$ y $w$ de longitud $k$ de los cuales $k$ símbolos son tomados de $v$ y los restantes de $u$ y $w$.
  Ejemplo tomando $n = 1$, $k = 8$ y $d = 3$.
  \begin{columns}
    \begin{column}{0.5\textwidth}
      \begin{small}

        $$0000000\; (0 \qquad 000 \; 0000 \; 1 \qquad 000) \: 00010$$
        $$0000000\; (1 \qquad 000 \; 0001 \; 0 \qquad 000) \: 00011$$
        $$\vdots$$
        $$0011111\; (0 \qquad 001 \; 1111 \; 1 \qquad 001) \: 10000$$
        $$0011111\; (1 \qquad 010 \; 0000 \; 0 \qquad 010) \: 00001$$
        $$\vdots$$
        $$1111110\; (1 \qquad 111 \; 1111 \; 0 \qquad 111) \: 11111$$
      \end{small}
    \end{column}
    \begin{column}{0.5\textwidth}  %%<--- here
    \end{column}
    \end{columns}
\end{frame}

\begin{frame}
  \frametitle{Análisis por caso - Caso 4}
  $$ x = u_{k-d-1+n} u_{k-d+n} \dots u_k \quad v_1 v_2 \dots v_k \quad w_1 w_2 \dots w_{d+1-n} $$
  con $n \in \{1, 2, \dots , d\}$

  Este caso representa cuando los $k + d + 1$ símbolos son tomados de tres palabras consecutivas $u$, $v$ y $w$ de longitud $k$ de los cuales $k$ símbolos son tomados de $v$ y los restantes de $u$ y $w$.
  Ejemplo tomando $\textbf{n = 2}$, $k = 8$ y $d = 3$.
  \begin{columns}
    \begin{column}{0.5\textwidth}
      \begin{small}

        $$000000\; (00 \qquad 00 \; 0000 \; 01 \qquad 00) \: 000010$$
$$000000\; (01 \qquad 00 \; 0000 \; 10 \qquad 00) \: 000011$$
$$\vdots$$
$$001111\; (01 \qquad 00 \; 1111 \; 10 \qquad 00) \: 111111$$
$$001111\; (10 \qquad 00 \; 1111 \; 11 \qquad 01) \: 000000$$
$$001111\; (11 \qquad 01 \; 0000 \; 00 \qquad 01) \: 000001$$
$$\vdots$$
$$111111\; (01 \qquad 11 \; 1111 \; 10 \qquad 11) \: 111111$$
      \end{small}
    \end{column}
    \begin{column}{0.5\textwidth}  %%<--- here
    \end{column}
    \end{columns}
\end{frame}

\begin{frame}
  \frametitle{Análisis por caso - Caso 4}
  $$ x = u_{k-d-1+n} u_{k-d+n} \dots u_k \quad v_1 v_2 \dots v_k \quad w_1 w_2 \dots w_{d+1-n} $$
  con $n \in \{1, 2, \dots , d\}$
  \begin{columns}
    \begin{column}{0.5\textwidth}

        $$000000\; (\textcolor{red}{00} \quad \textcolor{blue}{00} \; 0000 \; \textcolor{red}{01} \quad \textcolor{blue}{00}) \: 000010$$
        $$000000\; (\textcolor{red}{01} \quad \textcolor{blue}{00} \; 0000 \; \textcolor{red}{10} \quad \textcolor{blue}{00}) \: 000011$$
        $$\vdots$$
        $$001111\; (\textcolor{red}{01} \quad \textcolor{blue}{00} \; 1111 \; \textcolor{red}{10} \quad \textcolor{blue}{00}) \: 111111$$
        $$001111\; (\textcolor{cyan}{10} \quad \textcolor{blue}{00} \; \textcolor{magenta}{1111} \; \textcolor{red}{11} \quad \textcolor{gray}{01}) \: 000000$$
        $$001111\; (\textcolor{red}{11} \quad \textcolor{blue}{01} \; 0000 \; \textcolor{red}{00} \quad \textcolor{blue}{01}) \: 000001$$
        $$\vdots$$
        $$111111\; (\textcolor{red}{01} \quad \textcolor{blue}{11} \; 1111 \; \textcolor{red}{10} \quad \textcolor{blue}{11}) \: 111111$$
    \end{column}
    \begin{column}{0.5\textwidth}  %%<--- here
      \pause
      \begin{footnotesize}
        $$\underbrace{\quad \textcolor{red}{A} \quad }_{n} \quad \underbrace{\quad \textcolor{blue}{B} \quad }_{d + 1 - n}  \quad \underbrace{\quad C \quad }_{k-d+1} \quad \textcolor{red}{next(A)} \quad \textcolor{blue}{B}$$

        $$\underbrace{\quad \textcolor{cyan}{1\dots10} \quad }_{n} \; \underbrace{\quad \textcolor{blue}{B} \quad }_{d + 1 - n}  \; \underbrace{\; \textcolor{magenta}{1\dots1} \; }_{k-d+1} \quad \textcolor{red}{n(A)} \quad \textcolor{gray}{n(B)}$$
        
        Para el primer esquema tenemos $2^k \sum_{n=1}^{d} \frac{2^n - 1}{2^n}$

        Para el segundo esquema $ 2 \cdot 2^d \sum_{n=1}^{d} 2^{- n}$

        Juntos son menos que 
        $$d2^k + 2 \cdot 2^d$$
      \end{footnotesize}
    \end{column}
    \end{columns}
\end{frame}


\begin{frame}
  \frametitle{Hay demasiadas palabras que no aparecen en $champ$}
  Si $champ$ fuera supernormal, la frecuencia experada de palabras que aparecen una o más veces en los primeros $2^n$ símbolos debería satisfacer:
$$\lim_{n\to\infty} \frac{\#\{w: |w| = n  , |champ[1...2^n]|_w > 0\}}{2^n}  = 1 - e^{-1}$$
\pause
Sabemos que $X(k)$ ocupa la mitad de las palabras de los primeros $2^{d+k+1}$ símbolos de $champ$.
Si analizamos lo que sucede en las palabras de longitud $d+k+1$ usando las cotas que calculamos en los diferentes casos, y asumimos que la mitad restante ($2^{d+k} + d + k - 1$) son todas diferentes, entonces:
  $$\lim_{d\to\infty} \frac{\#\{w : |w| = d+k+1, |champ[1 \dots 2^{d+k+1}]|_w > 0 \}}{2^{d+k+1}} <$$
  $$\lim_{d\to\infty} \frac{ \textrm{Caso 1} + \textrm{Caso 2}+ \textrm{Caso 4} + \textrm{Caso 5} + \textrm{otra mitad}}{2^{d+k+1}} =$$

  \pause 

    $$\lim_{d\to\infty} \frac{(2^k - 2\cdot 2^d) + (2 \cdot 2^k)+ (d2^k + 2 \cdot 2^d) + (2d+k+1) + (2^{d+k}) }{2^{d+k+1}} =$$
    $$\frac{1}{2} < 1 - e^{-1}$$

\end{frame}

\begin{frame}
  \frametitle{Acotando palabras distintas en $champ$}
  Finalmente, si en $champ$ consideramos que los $n$'s tales que $n = d + 2^d + 1$ son una subsecuencia de $n=1,2,\dots$, podemos afirmar que si:
  $$\lim_{n\to\infty} \frac{\#\{w: |w| = n  , |champ[1...2^n]|_w > 0\}}{2^n}$$
  existe, este no es $1 - e^{-1}$.

\begin{corollary}
Si $x$ es normal, no necesariamente $x$ es supernormal.
\end{corollary}
\end{frame}

\begin{frame}

  \frametitle{Supermormal $\Rightarrow$ Normal - Idea de la demostración}
  \begin{itemize}
    \item Fijamos $\lambda = 1$ y veremos que si $x$ es 1-supernormal entonces es normal. 
    \item Para cada $\epsilon$, definimos un  $k_0$ tal que desde la cantidad $k_0$ haya $\frac{\epsilon}{2}$ palabras.
    \item Consideramos que una posición en $x[1\dots2^n ]$ es \textit{culpable} si la palabra que comienza en esa posición ocurre más de $k_0$ veces en $x[1\dots2^n ]$.
    \item Acotamos la cantidad de posiciones culpables en $x[1\dots2^n ]$ por $2^n\epsilon$
    \item Con esto, podemos fijar un $n$, un $\epsilon$ y una palabra $w$ y dar una cota superior de la cantidad de apariciones de $w$ en $x[1\dots2^n ]$.
    \item En el límite, esta cota es la esperada ($2^{-|w|}$)
    \item Por último utilizamos el Hot-spot Lemma para ver que $x$ es normal.
  \end{itemize}
 
\end{frame}

\begin{frame}
  \frametitle{El conjunto $Bad$}
  \begin{definition}
    Sea $Bad(n, w, \epsilon)$el conjunto de palabras binarias de longitud $n$ donde $w$ difiere de la frecuencia esperada por más que
    $n \epsilon$. Entonces,
    $$Bad(n, w, \epsilon)=\Big{\{}v\in \{0,1\}^n: \Big{|} |v|_w - n 2^{-|w|} \Big{|} \geq \epsilon n \Big{\} }$$
  \end{definition}
  
\end{frame}

\begin{frame}
  \frametitle{Teniendo en cuenta}
  Una vez acotadas las posiciones culpables, fijamos $ n, \epsilon$ y $w$ y damos una cota superior de la cantidad
  de ocurrencias de $w$ en $x[1 .. 2^n]$. Para eso, consideramos que:

\begin{enumerate}
\item Cada posición culpable tiene a lo sumo una ocurrencia de $w$
\item Cada bloque en $Bad(n, w, \epsilon)$ ocurre en $x[1.. 2^n]$ a lo sumo $k_0$ veces
\item Cada bloque en $Bad(n, w, \epsilon)$ tiene a lo sumo $n-|w|+1$ ocurrencias de $w$.
\item La secuencia $x[1..  2^n]$ puede ser dividida en a lo sumo $ 2^n/n$ bloques consecutivos de longitud $n$.  
\item Cada bloque de longitud $n$ que comienza en una posición no culpable,  y no está en $bad$ contiene a lo sumo $n 2^{-|w|} + \epsilon n $ ocurrencias de $w$.
\item Entre dos bloques consecutivos de longitud $n$ hay a lo sumo $|w|-1$ ocurrencias de $w$.
\end{enumerate}
\end{frame}

\begin{frame}
  \frametitle{Cota para ocurrencias de $w$}
  \begin{align*}
    |x[1.. 2^n]|_w \leq &
     \#\text{posiciones culpables} +
    \\& k_0 \#(\text{bloques malos}) (n-|w|+1)+
    \\& \#(\text{bloques no malos empezando}
    \\& \text{en posiciones no culpables}) (n 2^{-|w|} + \epsilon n)+
    \\&  \#(\text{interbloques})  (|w|-1)
    \\
    \leq& 2^n \epsilon + 
    \\& 2^{n+1} e^{-\epsilon^2 n/(6|w|)} (n-|w|+1)+
    \\&  ( 2^n /n) (n 2^{-|w|} + \epsilon n)+
    \\&  ( 2^n/n) (|w|-1)
    \end{align*}

    \pause
    Tomando el límite nos queda que\begin{align*}
      \lim_{n\to \infty}\frac{|x[1.. 2^n]|_w }{ 2^n}= {2^{-|w|}}
      \end{align*}
      
\end{frame}

\begin{frame}
  \frametitle{Hot Spot Lemma}

  \begin{definition}
    Una secuencia es normal si y solo si existe una constante $C$ tal que para infinitas longitudes $\ell$ y para todo $w$ en 
    $\{0,1\}^\ell$,
    $$\limsup_{N\to \infty}  \frac{|x[1.. N]|_w }{N} \leq C \frac{1}{2^{|w|}}$$
  \end{definition}
  
  Podemos usar el Hot Spot Lemma para concluir que la secuencia es normal.
      
\end{frame}

\begin{frame}
  \frametitle{Agradecimiento y Trabajo futuro}
  % Esta tesis fue realizada con la dirección conjunta de Ver\'onica Becher y Olivier Carton en el marco del Laboratoire International Associé SINFIN, Université Paris Diderot-CNRS/Universidad de Buenos Aires-CONICET)\\
  % \bigskip
  \begin{itemize}
    \item Demostrar que casi todas las palabras son supernormales
    \item Encontrar una palabra supernormal (Dar un algoritmo)
    \item Caracterizar la supernormalidad en términos de compresión
    % \item Relación entre discrepancia y supernormalidad
  \end{itemize}
\end{frame}

\begin{frame}
  \frametitle{¡Fin!}
  \begin{center}
  {\Huge ¿Preguntas?}
  \end{center}
\end{frame}
  
  
  \begin{frame}  
    \frametitle{Bibliografía}

  \begin{thebibliography}{1}
  
  \setlength{\parskip}{-0.5mm}
  {\tiny
  \bibitem{hotspot}
    {\sc Bailey, D.~H., and Misiurewicz, M.}
    \newblock A strong hot spot theorem.
    \newblock {\em Proceedings of the American Mathematical Society 134}, 9 (2006),
      2495--2501.
    
    \bibitem{BC2018}
    {\sc Becher, V., and Carton, O.}
    \newblock Normal numbers and computer science.
    \newblock In {\em Sequences, Groups, and Number Theory}, V.~Berthé and
      M.~Rigo, Eds. Springer International Publishing, Cham, 2018, pp.~233--269.
    
    \bibitem{BCC2019}
    {\sc Becher, V., Carton, O., and Cunningham, I.}
    \newblock Low discrepancy sequences failing poissonian pair correlations.
    \newblock {\em Archiv der Mathematik 113\/} (04 2019).
    
    \bibitem{Borel}
    {\sc Borel, {\'E}.}
    \newblock Les probabilit{\'e}s d{\'e}nombrables et leurs applications
      arithm{\'e}tiques.
    \newblock {\em Rendiconti del Circolo Matematico di Palermo (1884-1940) 27}, 1
      (Dec 1909), 247--271.
    
    \bibitem{bugeaud}
    {\sc Bugeaud, Y.}
    \newblock {\em Distribution Modulo One and {D}iophantine Approximation}.
    \newblock No.~193 in Cambridge Tracts in Mathematics. Cambridge University
      Press, Cambridge, UK, 2012.
    
    \bibitem{champern}
    {\sc Champernowne, D.~G.}
    \newblock {The Construction of Decimals Normal in the Scale of Ten}.
    \newblock {\em Journal of the London Mathematical Society s1-8}, 4 (10 1933),
      254--260.
    
    \bibitem{FN}
    {\sc Figuiera, S., and Nies, A.}
    \newblock Feasible analysis and randomness.
    \newblock {\em Theory Comput Syst (2015) 56\/} (2015), 439.
    
    \bibitem{kuipers}
    {\sc Kuipers, L., and Niederreiter, H.}
    \newblock {\em Uniform distribution of sequences}.
    \newblock Dover Publications, Inc., New York, 2006.
    
    \bibitem{PS2019}
    {\sc Pirsic, I., and Stockinger, W.}
    \newblock The champernowne constant is not poissonian.
    \newblock {\em Funct. Approx. Comment. Math. 60}, 2 (03 2019), 253--262.

    
    \bibitem{Sierpinski}
    {\sc Sierpinski, W.}
    \newblock D\'emonstration \'el\'ementaire du th\'eor\`eme de m. borel sur les
      nombres absolument normaux et d\'etermination effective d'une tel nombre.
    \newblock {\em Bulletin de la Soci\'et\'e Math\'ematique de France 45\/}
      (1917), 125--132.
    
    \bibitem{Weiss}
    {\sc Weiss, B.}
    \newblock Random-like behavior in deterministic systems -
      https://youtu.be/8ab7591de68?t=1280, 2010.
  } 
    \end{thebibliography}  
  \end{frame}


 
\end{document}


